


\usepackage[normalem]{ulem}

\makeatletter%
\@ifclassloaded{beamer}%
{
%----------------------------------------------------------
% glossário e acrônimos
%
\usepackage[acronym]{glossaries} %GLOSSÁRIO
%\GlsSetXdyCodePage{utf8}
\glsnoexpandfields
\glsaddall
\makeglossaries

\newglossarystyle{dotglos}{%
	\setglossarystyle{list}%
	\renewcommand*{\glossentry}[2]{%
		\item[\glsentryitem{##1}\glstarget{##1}{\glossentryname{##1}}]
		\ifglshassymbol{##1}{[\glossentrysymbol{##1}]\quad}{}%
		\emph{\glossentrydesc{##1}}%
		\unskip\leaders\hbox to 2.9mm{\hss.}\hfill##2}%
	\renewcommand*{\glsgroupskip}{}%
}

%\newglossarystyle{dotglos}{%
%	\setglossarystyle{list}% base this style on the list style
%	\renewcommand*{\glossentry}[2]{%
%		\item[\glsentryitem{##1}%
%		\glstarget{##1}{\glossentryname{##1}}]
%		\glossentrydesc{##1}\glspostdescription\space
%		\unskip\leaders\hbox to 2.9mm{\hss.}\hfill\ifnum\glsentryprevcount{##1}> pp.\else p.\fi\ ##2}%
%}

\setglossarystyle{dotglos}
%----------------------------------------------------------
}{}
\makeatother%


%----------------------------------------------------------
% alinhar texto justificado
%
\usepackage{ragged2e}
%----------------------------------------------------------





\newcommand\sbullet[1][.5]{\mathbin{\ThisStyle{\vcenter{\hbox{%
					\scalebox{#1}{$\SavedStyle\bullet$}}}}}%
}



% para codigos




\usepackage{listings}
\usepackage{listingsutf8}
\usepackage{inconsolata}

\definecolor{dkgreen}{rgb}{0,0.6,0}
\definecolor{gray}{rgb}{0.5,0.5,0.5}
\definecolor{mauve}{rgb}{0.58,0,0.82}

\newcommand{\indentrule}{\color{gray}\rlap{\smash{\hspace{6pt}\rule[-.35em]{1pt}{1.45em}}}}

\lstdefinestyle{CBruno}{
  inputencoding=utf8,
%   extendedchars=false,
    language=python,
    backgroundcolor=\color{white},
    commentstyle=\color{dkgreen},
    morestring=[s]{"""}{"""},
    % keywordstyle=\color{blue},
    % keywordstyle={[2]\color{magenta}},
    numberstyle=\tiny\color{gray},
    stringstyle=\color{mauve},
    % basicstyle=\footnotesize,
    basicstyle=\tiny,
    comment=[l]{\#},
    escapechar=@,
    escapeinside={\%*}{\text{\%}*)},
    % escapeinside={;@}{\^^M},
    otherkeywords={*,...},
    % escapeinside={\%*}{*)},
    % keywords={@relation,@attribute,@data},
    % morekeywords=[2]{real,integer,numeric,string,date},
    breakatwhitespace=true,
    breaklines=true,
    captionpos=b,
    keepspaces=true,
    firstnumber=1,
    numbers=left,
    numbersep=5pt,
    showspaces=false,
    showstringspaces=false,
    showtabs=false,
    tabsize=2,
    frame=single,
    rulecolor=\color{black},
    keywordstyle=\color{blue},
    % morekeywords={*,...},
    % alsoletter={., [\_]},
    texcl=true,
    alsoletter ={_},
    otherkeywords = {!,!=,~,$,*,\&,\%/\%,\%*\%,\%\%,<-,<<-},
    % morecomment=[s][\color{black}]{<!--}{-->},
    stepnumber=1,
    % morecomment=[l][]{//}, 
    % morecomment=[s][]{/*}{*/},
    % morestring=[b]",
    % morestring=[b]',
    extendedchars=true,
    literate=*
    {á}{{\'a}}1 {é}{{\'e}}1 {í}{{\'i}}1 {ó}{{\'o}}1 {ú}{{\'u}}1
    {Á}{{\'A}}1 {É}{{\'E}}1 {Í}{{\'I}}1 {Ó}{{\'O}}1 {Ú}{{\'U}}1
    {à}{{\`a}}1 {è}{{\`e}}1 {ì}{{\`i}}1 {ò}{{\`o}}1 {ù}{{\`u}}1
    {À}{{\`A}}1 {È}{{\'E}}1 {Ì}{{\`I}}1 {Ò}{{\`O}}1 {Ù}{{\`U}}1
    {ä}{{\"a}}1 {ë}{{\"e}}1 {ï}{{\"i}}1 {ö}{{\"o}}1 {ü}{{\"u}}1
    {Ä}{{\"A}}1 {Ë}{{\"E}}1 {Ï}{{\"I}}1 {Ö}{{\"O}}1 {Ü}{{\"U}}1
    {â}{{\^a}}1 {ê}{{\^e}}1 {î}{{\^i}}1 {ô}{{\^o}}1 {û}{{\^u}}1
    {ã}{{\~a}}1 {ẽ}{{\~e}}1 {ĩ}{{\~i}}1 {õ}{{\~o}}1 {ũ}{{\~u}}1
    {Â}{{\^A}}1 {Ê}{{\^E}}1 {Î}{{\^I}}1 {Ô}{{\^O}}1 {Û}{{\^U}}1
    {œ}{{\oe}}1 {Œ}{{\OE}}1 {æ}{{\ae}}1 {Æ}{{\AE}}1 {ß}{{\ss}}1
    {ç}{{\c c}}1 {Ç}{{\c C}}1 {ø}{{\o}}1 {å}{{\r a}}1 {Å}{{\r A}}1
    {€}{{\EUR}}1 {£}{{\pounds}}1 {^}{\text{\^{}}}1 {\\}{{$\textbackslash$}}1
    {\%}{{\%}}1 
}

\lstset{style=CBruno}

% https://www.cl.uni-heidelberg.de/courses/ss19/wissschreib/material/LaTeX.pdf







\makeatletter%
\@ifclassloaded{beamer}%
{
    \setbeamercolor{saibamais}{fg=cinzaescuro,bg=cinzaclaro}
	\setbeamercolor{saibamaistitulo}{fg=aliceblue,bg=cinzaescuro}
	\setbeamercovered{transparent}
	\renewcommand{\inputFilesStartValue}{1}
	\renewcommand{\inputFilesMaxValue}{24}
}{}
\makeatother%

%----------------------------------------------
% VARIÁVEIS DO USUÁRIO
%----------------------------------------------
\titulo{Backend}
\disciplina{Python com Django}
\turma{}
\autor{Prof. Bruno Iran Ferreira Maciel}
\cargahoraria{}
\alocacao{Turma 19}
\local{RECIFE}
\data{\mydate\today}
\instituicao{}%Faculdade de Ciências Humanas ESUDA}
\nomecurso{}
\programa{}%Graduação em Sistemas de Informação}
% \emailprograma{posgraduacao@cin.ufpe.br}
% \siteprograma{http://cin.ufpe.br/\textasciitilde posgraduacao}

\siteprograma{\href{http://brunomaciel.com}{\beamerbutton{brunomaciel.com}}}%ESUDA}
\definecolor{cinColor}{HTML}{e07c40} %cor padrão

\figuralogo{imagens/logo.png}
\figuralogocapa{imagens/logo-capa.png}
\figuradmca{imagens/fig-dmca.png}

\pgfplotstableread[col sep=&,header=true]{
N & Aulas & Mês & Data & Conteúdo Previsto
1 & 1,5 & Agosto & 02/08/2024 & Apresentação e boas-vindas
2 & 3 & Agosto & 02/08/2024 & Lógica de Prog. e Padrões de Des. de Software

3 & 4,5 & Agosto & 03/08/2024 & Lógica de Programação
4 & 6 & Agosto & 03/08/2024 & Padrões de Desenvolvimento de Software

5 & 7,5 & Agosto & 09/08/2024 & Lógica de Programação
6 & 9 & Agosto & 09/08/2024 & Padrões de Desenvolvimento de Software

7 & 10,5 & Agosto & 10/08/2024 & POO
8 & 12 & Agosto & 10/08/2024 & Padrões de Desenvolvimento de Software

9 & 13,5 & Agosto & 16/08/2024 & POO
10 & 15 & Agosto & 16/08/2024 & Git

11 & 16,5 & Agosto & 17/08/2024 & POO
12 & 18 & Agosto & 17/08/2024 & Padrões de Desenvolvimento de Software

13 & 19,5 & Agosto & 23/08/2024 & POO
14 & 21 & Agosto & 23/08/2024 & Padrões de Desenvolvimento de Software

15 & 22,5 & Agosto & 24/08/2024 & POO
16 & 24 & Agosto & 24/08/2024 & Padrões de Desenvolvimento de Software

17 & 25,5 & Agosto & 30/08/2024 & POO
18 & 27 & Agosto & 30/08/2024 & Padrões de Desenvolvimento de Software

19 & 28,5 & Agosto & 31/08/2024 & POO
20 & 30 & Agosto & 31/08/2024 & Soft Skills

21 & 31,5 & Setembro & 06/09/2024 & POO
22 & 33 & Setembro & 06/09/2024 & Padrões de Desenvolvimento de Software

23 & 0 & Setembro & 07/09/2024 & Feriado Nacional - Independência do Brasil
24 & 0 & Setembro & 07/09/2024 & Feriado Nacional - Independência do Brasil

x &  & Setembro & 13/09/2024 & POO
x &  & Setembro & 13/09/2024 & Padrões de Desenvolvimento de Software

x &  & Setembro & 14/09/2024 & POO
x &  & Setembro & 14/09/2024 & Padrões de Desenvolvimento de Software

x &  & Setembro & 20/09/2024 & POO
x &  & Setembro & 20/09/2024 & Padrões de Desenvolvimento de Software

x &  & Setembro & 21/09/2024 & POO
x &  & Setembro & 21/09/2024 & Padrões de Desenvolvimento de Software

x &  & Setembro & 27/09/2024 & POO
x &  & Setembro & 27/09/2024 & Padrões de Desenvolvimento de Software

x &  & Setembro & 28/09/2024 & POO
x &  & Setembro & 28/09/2024 & Padrões de Desenvolvimento de Software

x &  & Outubro & 04/10/2024 & POO
x &  & Outubro & 04/10/2024 & Django

x &  & Outubro & 05/10/2024 & Soft Skills
x &  & Outubro & 05/10/2024 & POO

x &  & Outubro & 11/10/2024 & Django
x &  & Outubro & 11/10/2024 & Web Services

x &  & Outubro & 12/10/2024 & Feriado Nacional - Nossa Senhora Aparecida

x &  & Outubro & 18/10/2024 & Django
x &  & Outubro & 18/10/2024 & Web Services

x &  & Outubro & 19/10/2024 & Django
x &  & Outubro & 19/10/2024 & Web Services

x &  & Outubro & 25/10/2024 & Django
x &  & Outubro & 25/10/2024 & Web Services

x &  & Outubro & 26/10/2024 & Django
x &  & Outubro & 26/10/2024 & Web Services

x &  & Novembro & 01/11/2024 & Django
x &  & Novembro & 01/11/2024 & Web Services

x &  & Novembro & 02/11/2024 & Feriado Nacional - Finados

x &  & Novembro & 08/11/2024 & Django
x &  & Novembro & 08/11/2024 & Web Services

x &  & Novembro & 09/11/2024 & Soft Skills
x &  & Novembro & 09/11/2024 & Django

x &  & Novembro & 15/11/2024 & Feriado Nacional - Proclamação da República
x &  & Novembro & 16/11/2024 & Imprensado - Não teremos aulas

x &  & Novembro & 22/11/2024 & Django
x &  & Novembro & 22/11/2024 & Web Services

x &  & Novembro & 23/11/2024 & Django
x &  & Novembro & 23/11/2024 & Web Services
 
x &  & Novembro & 29/11/2024 & Django
x &  & Novembro & 29/11/2024 & Web Services

x &  & Novembro & 30/11/2024 & Django
x &  & Novembro & 30/11/2024 & Web Services


}\cronograma

%: hadoop (HDFS), mapreduce, googlefs, gusterfs, DNS, bonding, apache, pacemaker, DRDB; openstack}
\newcommand{\columnIndex}{4}


% para suportar icones como checkmark
\usepackage{bbding}
%\Checkmark
%\CheckmarkBold
%\XSolid
%\XSolidBold
%\XSolidBrush


% criar uml de classes
\usetikzlibrary{shapes}
\definecolor{bottomcol}{RGB}{222,222,222}
\usetikzlibrary{positioning} %requer esta biblioteca
%\tikzset{
%	reduce height/.style={
%		minimum height=0pt,
%		inner ysep=#1,
%		text depth=2pt
%	},
%	reduce height/.default={0pt}
%}

% é preciso instalar um pacote do pyhton por fora. Use o comando para instalar > pip install Pygments
\usepackage[newfloat]{minted}
\setminted{
	breaklines=true,
	encoding=utf8,
	fontseries=heiti,
	% autogobble=true,
	frame=lines, % Adiciona uma borda ao redor do código
	framesep=2mm, % Espaçamento entre a borda e o código
	baselinestretch=0.75, % Espaçamento entre linhas
	bgcolor=bg, % Cor de fundo
%	fontsize=\footnotesize, % Tamanho da fonte
	fontsize={\fontsize{5.5}{6.5}\selectfont},
	linenos % Adiciona números de linha
}   

% learned from issue #238 #69 
\makeatletter
\AtBeginEnvironment{minted}{\dontdocolorbox}
\def\dontdofcolorbox{\renewcommand\fcolorbox[4][]{##4}}
\AtBeginEnvironment{minted}{\renewcommand{\colorbox}[3][]{#3}}
\makeatother

% some other elements


\newenvironment{code}{\captionsetup{type=listing}}{}
\SetupFloatingEnvironment{listing}{name=Código}


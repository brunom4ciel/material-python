

% para contar as aparições
\glsenableentrycount


\newglossaryentry{exercicio_001}
{
	name={\textit{Exercício 001}},
	description={Ler 4 valores (considere que não serão informados valores iguais). Escreva a soma dos dois últimos números.},
	plural={}
}



\newglossaryentry{exercicio_002}
{
	name={\textit{Exercício 002}},
	description={Ler 2 valores e se o segundo valor informado for ZERO, deve ser lido um novo valor, ou seja, para o segundo valor não pode ser aceito o valor zero e imprimir o resultado da divisão do primeiro valor lido pelo segundo valor lido. (utilizar a estrutura REPETIR)},
	plural={}
}


\newglossaryentry{exercicio_003}
{
	name={\textit{Exercício 003}},
	description={Ler as idades de 2 homens e de 2 mulheres (considere que as idades dos homens serão sempre diferentes entre si, bem como as das mulheres). Calcule e escreva a soma das idades do homem mais velho com a mulher mais nova, e o produto das idades do homem mais novo com a mulher mais velha.},
	plural={}
}



\newglossaryentry{exercicio_004}
{
	name={\textit{Exercício 004}},
	description={Ler o salário fixo e o valor total das vendas efetuadas pelo vendedor de uma empresa. Sabendo-se que ele recebe uma comissão de 3\% sobre o total das vendas até R\$ 1.500,00 mais 5\% sobre  o que ultrapassar este valor, calcular e escrever o seu salário total.},
	plural={}
}

\newglossaryentry{exercicio_005}
{
	name={\textit{Exercício 005}},
	description={Ler 11 valores numéricos, somar os 10 primeiros e guardar em uma variável A e o décimo	primeiro valor, guardar em uma variável B. Escreva os valores de A e B. A seguir (utilizando apenas atribuições entre variáveis) troque os seus conteúdos fazendo com que o valor que está em A passe para B e vice-versa. Ao final, escreva os valores que ficaram	armazenados nas variáveis.},
	plural={}
}

\newglossaryentry{exercicio_006}
{
	name={\textit{Exercício 006}},
	description={Ler um valor numérico e escrever o seu antecessor. Ex: Ler n = 20, Escreva 19.},
	plural={}
}


\newglossaryentry{exercicio_007}
{
	name={\textit{Exercício 007}},
	description={Ler três valores que representam a idade de uma pessoa, expressa em anos, meses e dias (data de nascimento).	Escreva a idade dessa pessoa expressa apenas em dias. Considerar ano com 365 dias e mês	com 30 dias.},
	plural={}
}


\newglossaryentry{exercicio_008}
{
	name={\textit{Exercício 008}},
	description={Ler o número total alunos de uma sala de aula, o número de votos em candidato A e candidato B. Escreva o percentual que cada candidato representa em relação ao total de	alunos. Considere que o número total de alunos votou no candidato A ou B.},
	plural={}
}



\newglossaryentry{exercicio_009}
{
	name={\textit{Exercício 009}},
	description={Sistema de ordenação de valores. Ler 5 valores (considere que não serão informados valores iguais). Escrever os números em ordem CRESCENTE.},
	plural={}
}



\newglossaryentry{exercicio_010}
{
	name={\textit{Exercício 010}},
	description={Sistema de ordenação de valores. Ler 5 valores (considere que não serão informados valores iguais). Escrever os números em ordem DECRESCENTE.},
	plural={}
}


\newglossaryentry{exercicio_011}
{
	name={\textit{Exercício 011}},
	description={Ler x números, onde x é definido pelo usuário (o usuário que decide quando acaba). Escreva o resultado da subtração entre as somas dos números pares e ímpares. Ex: soma dos pares - soma dos ímpares.},
	plural={}
}

\newglossaryentry{exercicio_012}
{
	name={\textit{Exercício 012}},
	description={Ler 3 valores e não aceitar valores menores que 1. Caso o usuário digite valor menor que 1, repetir até obter todos os números. Escreva o resultado da soma dos números.},
	plural={}
}


% novos 2024-08-07


\newglossaryentry{exercicio_013}
{
	name={\textit{Exercício 013}},
	description={Leia três números inteiros e calcule a soma. Considerar que a condição, se a soma for maior que 10, escreva “tem erro”, do contrário escreva o valor resultante da soma.},
	plural={}
}



\newglossaryentry{exercicio_014}
{
	name={\textit{Exercício 014}},
	description={Leia três notas de um aluno, calcule e escreva a média final deste
		aluno. Considerar que a média é ponderada e que o peso das notas são 2 para a primeira nota, 3 para a segunda nota e 5 para a última nota.},
	plural={}
}




\newglossaryentry{exercicio_015}
{
	name={\textit{Exercício 015}},
	description={Leia o número de maçãs compradas, calcule e escreva o custo total da compra. Considere que as maçãs custam R\$ 1,50 cada se forem compradas menos de uma dúzia, e R\$ 1,00 se forem compradas pelo menos 12.},
	plural={}
}


\newglossaryentry{exercicio_016}
{
	name={\textit{Exercício 016}},
	description={A jornada de trabalho semanal de um funcionário é de 40 horas. O funcionário que trabalhar mais de 40 horas receberá hora extra, cujo cálculo é o valor da hora regular com um acréscimo de 50\%. Leia o número de horas trabalhadas em um mês, o salário por hora e escreva o salário total do funcionário, que deverá ser acrescido das horas extras, caso tenham sido trabalhadas (considere que o mês possua 4 semanas exatas).},
	plural={}
}



\newglossaryentry{exercicio_017}
{
	name={\textit{Exercício 017}},
	description={Ler 4 números inteiros que correspondem ao número da conta do cliente, saldo, débito ou crédito. Os número serão passados na inicialização do script. Calcular e escrever o saldo atual (saldo atual = saldo - débito + crédito). Também verificar se saldo atual for maior ou igual a zero, escrever a mensagem 'Saldo Positivo', senão escrever a mensagem 'Saldo Negativo'.},
	plural={}
}


\newglossaryentry{exercicio_018}
{
	name={\textit{Exercício 018}},
	description={Ler 3 números inteiros que correspondem a (1) quantidade atual em estoque, (2) quantidade máxima em estoque e (3) quantidade mínima em estoque de um produto. Calcular e escrever a quantidade média ((quantidade média = quantidade máxima + quantidade mínima)/2). Se a quantidade em estoque for maior ou igual a quantidade média escrever a mensagem 'Não efetuar compra', senão escrever a mensagem 'Efetuar compra'.},
	plural={}
}



\newglossaryentry{exercicio_019}
{
	name={\textit{Exercício 019}},
	description={Ler um valor e escrever se é positivo, negativo ou zero.},
	plural={}
}



\newglossaryentry{exercicio_020}
{
	name={\textit{Exercício 020}},
	description={Ler 3 valores (considere que não serão informados valores iguais) e escrever o maior deles.},
	plural={}
}


\newglossaryentry{exercicio_021}
{
	name={\textit{Exercício 021}},
	description={Ler 3 valores (A, B e C) representando as medidas dos lados de um triângulo e escrever se formam ou não um triângulo. OBS: para formar um triângulo, o valor de cada lado deve ser menor que a soma dos outros 2 lados.},
	plural={}
}


\newglossaryentry{exercicio_022}
{
	name={\textit{Exercício 022}},
	description={Ler o nome de 2 times e o número de gols marcados na partida (para cada time). Escrever o nome do vencedor. Caso não haja vencedor deverá ser impressa a palavra EMPATE.},
	plural={}
}


\newglossaryentry{exercicio_023}
{
	name={\textit{Exercício 023}},
	description={Ler dois valores e imprimir uma das três mensagens a seguir:\\
		‘Números iguais’, caso os números sejam iguais;\\
		‘Primeiro é maior’, caso o primeiro seja maior que o segundo;\\
		‘Segundo maior’, caso o segundo seja maior que o primeiro.},
	plural={}
}


\newglossaryentry{exercicio_024}
{
	name={\textit{Exercício 024}},
	description={Ler uma string que contém o endereço de um arquivo em disco local. Exemplo de string = \"c:\\users\\biblioteca\\pessoal\\documentos\\rg\\rg-bruno.pdf\". Retorne o nome do arquivo com extensão.},
	plural={}
}


\newglossaryentry{exercicio_025}
{
	name={\textit{Exercício 025}},
	description={Leia um número inteiro. Escreva o número lido.},
	plural={}
}

\newglossaryentry{exercicio_026}
{
	name={\textit{Exercício 026}},
	description={Leia três números inteiro e guarde em uma lista. Escreva os números da lista.},
	plural={}
}


\newglossaryentry{exercicio_027}
{
	name={\textit{Exercício 027}},
	description={Leia trinta números inteiro e guarde em uma lista. Escreva os números da lista.},
	plural={}
}


\newglossaryentry{exercicio_028}
{
	name={\textit{Exercício 028}},
	description={Leia n (assuma como premissa que n = 90) números inteiro e guarde em uma lista. Escreva os números da lista.},
	plural={}
}



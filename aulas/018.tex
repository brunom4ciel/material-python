\pgfplotstablegetelem{\thepart}{[index]\columnIndex}\of{\cronograma}
\part{\pgfplotsretval}
\label{part:\thepart}
\frame{\partpage}


\begin{frame}[t]{Padrões de desenvolvimento de software}
	
	\fontsize{12pt}{15}\selectfont{
		...continuando Padrões de desenvolvimento de software.\\ 
	}\par
	\vspace{1em}
	
	
	\fontsize{12pt}{15}\selectfont{
		\begin{itemize}%[<+->]  
			
			\item {\color{red}Padrões de Criação.}
			
			\begin{itemize}%[<+->]
				\item {\color{blue}Singleton \CheckmarkBold}
				\item {\color{blue}Abstract Factory \CheckmarkBold}
				\item {\color{blue}Factory Method \CheckmarkBold}
				\item Builder \XSolidBrush
				\item Prototype \XSolidBrush
			\end{itemize}
			
		\end{itemize}
	}\par
	\vspace{1em}
	
	
\end{frame}





\begin{frame}[t]{Padrões de criação - Builder}
	\fontsize{12pt}{15}\selectfont{
		{\color{blue}Builder (Construtor - 'de construir coisas') \CheckmarkBold}
	}\par
	\vspace{1em}
	
	\fontsize{12pt}{15}\selectfont{
		\begin{itemize}%[<+->]  
			
			\item Padrão de criação que se concentra em como construir objetos complexos de maneira controlada e eficiente.
			
			\item \textbf{Ele separa a construção de um objeto da sua representação final, permitindo a criação de diferentes representações ou configurações do mesmo objeto.}
			
			\item Esse padrão é especialmente útil quando um objeto precisa ser construído passo a passo, ou quando o processo de construção é muito complexo.
			
		\end{itemize}
	}\par
	\vspace{1em}
\end{frame}



\begin{frame}[t]{Padrões de criação - Builder}
	\fontsize{12pt}{15}\selectfont{
		Exemplo
	}\par
	\vspace{0.5em}
	
	\fontsize{12pt}{15}\selectfont{
		\begin{enumerate}%[<+->]
			
			\item \textbf{Classe ProdutoBase} - Será o objeto complexo que queremos construir. Ele pode ter várias partes diferentes que são configuradas durante o processo de construção.
			
			\item \textbf{Classe BuilderAbstrata} - É uma interface ou classe abstrata que define os métodos para criar as diferentes partes do Produto.
			
			\item \textbf{Classe BuilderConcreta} - Implementa a interface do Builder e constrói as partes específicas do Produto.
			
			\item \textbf{Classe Diretor} - define a ordem de construção das partes do Produto. Ele usa o Builder para construir o Produto passo a passo.
			
			\item \textbf{Código do cliente para utilizar o Padrão Builder} - Quem usa o Diretor e o Builder para construir os objetos complexos.
			
		\end{enumerate}
	}\par
	\vspace{1em}
\end{frame}




\begin{frame}[t]{Padrões de criação - Builder}
	\fontsize{14pt}{15}\selectfont{
		Resumo
	}\par
	\vspace{1em}
	
	\fontsize{12pt}{15}\selectfont{
		\begin{itemize}%[<+->]
			
			\item Produto: O objeto que está sendo construído.
			
			\item Builder: Interface que define os métodos de construção.
			
			\item BuilderConcreta: Implementa os métodos do Builder e constrói as partes do Produto.
			
			\item Diretor: Controla o processo de construção, definindo a ordem em que as partes são construídas.
			
		\end{itemize}
	}\par
	\vspace{1em}
\end{frame}



\begin{frame}[t]{Padrões de criação - Builder}
	\fontsize{16pt}{15}\selectfont{
		Aplicação
	}\par
	\vspace{1em}

	\fontsize{14pt}{15}\selectfont{
		\begin{itemize}%[<+->]

			\item O padrão Builder é útil quando queremos construir objetos que exigem uma criação passo a passo, ou quando diferentes representações do mesmo objeto podem ser necessárias.

		\end{itemize}
	}\par
	\vspace{1em}
\end{frame}




\begin{frame}[t]{Padrões de criação - Builder}

	\vspace{1em}
	\begin{tikzpicture}[
		every node/.style={
			draw=blue!50!white,
			minimum width=2.5in,
			minimum height=1.5em,
			font=\sffamily,
			anchor=north,
		},
		connect/.style={
			draw=blue!70!white,
			-stealth,
		},
		]
		
		\node[draw] (BuilderAbstrata) at (0,0) {Class BuilderAbstrata}; %nó A
		\node[below=of BuilderAbstrata] (BuilderConcreta) {Class BuilderConcreta}; %nó B
		\draw[<-] (BuilderAbstrata) to (BuilderConcreta);
		
		\node[below=of BuilderConcreta] (Diretor) {Class Diretor}; %nó B
		\draw[<-] (BuilderConcreta) to (Diretor);
		
		\node[right=of Diretor] (ProdutoBase) {Class ProdutoBase}; %nó B
		\draw[->] (BuilderConcreta) to (ProdutoBase);
		
		\node[below=of Diretor] (codigoClienteBuilder) {def codigo\_cliente\_builder}; %nó B
		\draw[<-] (Diretor) to (codigoClienteBuilder);
		\draw[<-] (BuilderConcreta) to[out=0,in=-80]  (codigoClienteBuilder);
		
	\end{tikzpicture}

\end{frame}




\begin{frame}[t]{Padrões de criação - Builder}
	

	\centering
	\begin{minipage}{4cm}
		\begin{block}{}
			Class ProdutoBase
		\end{block}
	\end{minipage} 
	$\underrightarrow{\makebox[2cm][r]{Possui}\hspace{2em}}$
	\begin{minipage}{6cm}
		\begin{block}{}
			Lógica do produto.
		\end{block}
	\end{minipage}
	\vfill
	\begin{minipage}{4cm}
		\begin{block}{}
			Class BuilderAbstrata
		\end{block}
	\end{minipage} 
	$\underrightarrow{\makebox[2cm][r]{Possui}\hspace{2em}}$
	\begin{minipage}{6cm}
		\begin{block}{}
			Interface para definir a construção do produto.
		\end{block}
	\end{minipage}
	\vfill
	\begin{minipage}{4cm}
		\begin{block}{}
			Class BuilderConcreta
		\end{block}
	\end{minipage} 
	$\underrightarrow{\makebox[2cm][r]{Possui}\hspace{2em}}$
	\begin{minipage}{6cm}
		\begin{block}{}
			Implementa a lógica para definir a construção do produto.
		\end{block}
	\end{minipage}
	\vfill
	\begin{minipage}{4cm}
		\begin{block}{}
			Class DiretorConcreta
		\end{block}
	\end{minipage} 
	$\underrightarrow{\makebox[2cm][r]{Possui}\hspace{2em}}$
	\begin{minipage}{6cm}
		\begin{block}{}
			Implementa a lógica para definir a ordem de construção do produto.
		\end{block}
	\end{minipage}
	\vfill
	\begin{minipage}{4cm}
		\begin{block}{}
			def codigo\_cliente\_builder
		\end{block}
	\end{minipage} 
	$\underrightarrow{\makebox[2cm][r]{Possui}\hspace{2em}}$
	\begin{minipage}{6cm}
		\begin{block}{}
			Implementa o código do cliente para obter o produto final.
		\end{block}
	\end{minipage}
\end{frame}





\begin{frame}[t]{Padrão de projeto - Builder}

%	\lstinputlisting[style=CBruno,caption=Código da Classe Produto Base - codigo\_017\_produto\_base.py]{outros/codigos/python/exemplos-de-aulas/src/padroesdeprojetos/criacao/codigo_017_produto_base.py}
	
	\centering
	\makebox[\linewidth][c]{
		\begin{minipage}{0.95\textwidth}
			\inputminted[fontsize={\fontsize{10}{8}\selectfont}]{python}{outros/codigos/python/exemplos-de-aulas/src/padroesdeprojetos/criacao/codigo_017_produto_base.py}
		\end{minipage}
	}
	\fontsize{7pt}{6}\selectfont{
		Código da Classe Produto Base - codigo\_017\_produto\_base.py
	}\par

\end{frame}

\begin{frame}[t]{Padrão de projeto - Builder}

%	\lstinputlisting[style=CBruno,caption=Código da Classe Builder Abtratato - codigo\_017\_builder\_abstrato.py]{outros/codigos/python/exemplos-de-aulas/src/padroesdeprojetos/criacao/codigo_017_builder_abstrato.py}
	
	\centering
	\makebox[\linewidth][c]{
		\begin{minipage}{0.95\textwidth}
			\inputminted[fontsize={\fontsize{8}{6}\selectfont}]{python}{outros/codigos/python/exemplos-de-aulas/src/padroesdeprojetos/criacao/codigo_017_builder_abstrato.py}
		\end{minipage}
	}
	\fontsize{7pt}{6}\selectfont{
		Código da Classe Builder Abtratato - codigo\_017\_builder\_abstrato.py
	}\par
	
\end{frame}



\begin{frame}[t]{Padrão de projeto - Builder}

%	\lstinputlisting[style=CBruno,caption=Código da Classe Builder Concreta - codigo\_017\_builder\_concreta.py]{outros/codigos/python/exemplos-de-aulas/src/padroesdeprojetos/criacao/codigo_017_builder_concreta.py}
	
	\centering
	\makebox[\linewidth][c]{
		\begin{minipage}{0.95\textwidth}
			\inputminted[fontsize={\fontsize{6}{6}\selectfont}]{python}{outros/codigos/python/exemplos-de-aulas/src/padroesdeprojetos/criacao/codigo_017_builder_concreta.py}
		\end{minipage}
	}
	\fontsize{7pt}{6}\selectfont{
		Código da Classe Builder Concreta - codigo\_017\_builder\_concreta.py
	}\par
	

\end{frame}


\begin{frame}[t]{Padrão de projeto - Builder}
	
%	\lstinputlisting[style=CBruno,caption=Código da Classe Diretor Concreta - codigo\_017\_diretor\_base.py]{outros/codigos/python/exemplos-de-aulas/src/padroesdeprojetos/criacao/codigo_017_diretor_base.py}
	
	\centering
	\makebox[\linewidth][c]{
		\begin{minipage}{0.95\textwidth}
			\inputminted[baselinestretch=1,fontsize={\fontsize{8}{7}\selectfont}]{python}{outros/codigos/python/exemplos-de-aulas/src/padroesdeprojetos/criacao/codigo_017_diretor_base.py}
		\end{minipage}
	}
	\fontsize{7pt}{6}\selectfont{
		Código da Classe Diretor Concreta - codigo\_017\_diretor\_base.py
	}\par
	
\end{frame}


\begin{frame}[t]{Padrão de projeto - Builder}

%	\lstinputlisting[style=CBruno,caption=Código da Função Código do cliente - codigo\_017\_codigo\_cliente\_builder.py]{outros/codigos/python/exemplos-de-aulas/src/padroesdeprojetos/criacao/codigo_017_codigo_cliente_builder.py}

	\centering
	\makebox[\linewidth][c]{
		\begin{minipage}{0.95\textwidth}
			\inputminted[baselinestretch=1.25,fontsize={\fontsize{7.5}{6}\selectfont}]{python}{outros/codigos/python/exemplos-de-aulas/src/padroesdeprojetos/criacao/codigo_017_codigo_cliente_builder.py}
		\end{minipage}
	}
	\fontsize{7pt}{6}\selectfont{
		Código da Função Código do cliente - codigo\_017\_codigo\_cliente\_builder.py
	}\par
	
\end{frame}



\begin{frame}[t]{Pytest}

%	\lstinputlisting[style=CBruno,caption=Código do teste para o Código do cliente - test\_codigo\_017\_codigo\_cliente\_builder.py]{outros/codigos/python/exemplos-de-aulas/tests/test_codigo_017_codigo_cliente_builder.py}

	\centering
	\makebox[\linewidth][c]{
		\begin{minipage}{0.95\textwidth}
			\inputminted[baselinestretch=1.25,fontsize={\fontsize{10}{10}\selectfont}]{python}{outros/codigos/python/exemplos-de-aulas/tests/test_codigo_017_codigo_cliente_builder.py}
		\end{minipage}
	}
	\fontsize{7pt}{6}\selectfont{
		Código do teste para o Código do cliente - test\_codigo\_017\_codigo\_cliente\_builder.py
	}\par
	
\end{frame}






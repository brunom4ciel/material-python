\pgfplotstablegetelem{\thepart}{[index]\columnIndex}\of{\cronograma}
\part{\pgfplotsretval}
\label{part:\thepart}
\frame{\partpage}


\begin{frame}[t]{Padrões de desenvolvimento de software}
	
	\fontsize{12pt}{15}\selectfont{
		...continuando Padrões de desenvolvimento de software.\\ 
	}\par
	\vspace{1em}
	
	
	\fontsize{12pt}{15}\selectfont{
		\begin{itemize}%[<+->]  
			
			\item {\color{red}Padrões de Criação.}
			
			\begin{itemize}%[<+->]
				\item {\color{blue}Singleton \CheckmarkBold}
				\item {\color{blue}Abstract Factory \CheckmarkBold}
				\item {\color{blue}Factory Method \CheckmarkBold}
				\item Builder \XSolidBrush
				\item Prototype \XSolidBrush
			\end{itemize}
			
		\end{itemize}
	}\par
	\vspace{1em}
	
	
\end{frame}





\begin{frame}[t]{Padrões de criação - Builder}
	\fontsize{12pt}{15}\selectfont{
		{\color{blue}Builder (Construtor - 'de construir coisas') \CheckmarkBold}
	}\par
	\vspace{1em}
	
	\fontsize{12pt}{15}\selectfont{
		\begin{itemize}%[<+->]  
			
			\item Padrão de criação que se concentra em como construir objetos complexos de maneira controlada e eficiente.
			
			\item \textbf{Ele separa a construção de um objeto da sua representação final, permitindo a criação de diferentes representações ou configurações do mesmo objeto.}
			
			\item Esse padrão é especialmente útil quando um objeto precisa ser construído passo a passo, ou quando o processo de construção é muito complexo.
			
		\end{itemize}
	}\par
	\vspace{1em}
\end{frame}



\begin{frame}[t]{Padrões de criação - Builder}
	\fontsize{12pt}{15}\selectfont{
		Exemplo
	}\par
	\vspace{0.5em}
	
	\fontsize{12pt}{15}\selectfont{
		\begin{enumerate}%[<+->]
			
			\item \textbf{Classe ProdutoBase} - Será o objeto complexo que queremos construir. Ele pode ter várias partes diferentes que são configuradas durante o processo de construção.
			
			\item \textbf{Classe BuilderAbstrata} - É uma interface ou classe abstrata que define os métodos para criar as diferentes partes do Produto.
			
			\item \textbf{Classe BuilderConcreta} - Implementa a interface do Builder e constrói as partes específicas do Produto.
			
			\item \textbf{Classe Diretor} - define a ordem de construção das partes do Produto. Ele usa o Builder para construir o Produto passo a passo.
			
			\item \textbf{Código do cliente para utilizar o Padrão Builder} - Quem usa o Diretor e o Builder para construir os objetos complexos.
			
		\end{enumerate}
	}\par
	\vspace{1em}
\end{frame}




\begin{frame}[t]{Padrões de criação - Builder}
	\fontsize{14pt}{15}\selectfont{
		Resumo
	}\par
	\vspace{1em}
	
	\fontsize{12pt}{15}\selectfont{
		\begin{itemize}%[<+->]
			
			\item Produto: O objeto que está sendo construído.
			
			\item Builder: Interface que define os métodos de construção.
			
			\item BuilderConcreta: Implementa os métodos do Builder e constrói as partes do Produto.
			
			\item Diretor: Controla o processo de construção, definindo a ordem em que as partes são construídas.
			
		\end{itemize}
	}\par
	\vspace{1em}
\end{frame}



\begin{frame}[t]{Padrões de criação - Builder}
	\fontsize{16pt}{15}\selectfont{
		Aplicação
	}\par
	\vspace{1em}

	\fontsize{14pt}{15}\selectfont{
		\begin{itemize}%[<+->]

			\item O padrão Builder é útil quando queremos construir objetos que exigem uma criação passo a passo, ou quando diferentes representações do mesmo objeto podem ser necessárias.

		\end{itemize}
	}\par
	\vspace{1em}
\end{frame}




\begin{frame}[t]{Padrões de criação - Builder}

	\vspace{1em}
	\begin{tikzpicture}[
		every node/.style={
			draw=blue!50!white,
			minimum width=2.5in,
			minimum height=1.5em,
			font=\sffamily,
			anchor=north,
		},
		connect/.style={
			draw=blue!70!white,
			-stealth,
		},
		]
		
		\node[draw] (BuilderAbstrata) at (0,0) {Class BuilderAbstrata}; %nó A
		\node[below=of BuilderAbstrata] (BuilderConcreta) {Class BuilderConcreta}; %nó B
		\draw[<-] (BuilderAbstrata) to (BuilderConcreta);
		
		\node[below=of BuilderConcreta] (Diretor) {Class Diretor}; %nó B
		\draw[<-] (BuilderConcreta) to (Diretor);
		
		\node[right=of Diretor] (ProdutoBase) {Class ProdutoBase}; %nó B
		\draw[->] (BuilderConcreta) to (ProdutoBase);
		
		\node[below=of Diretor] (codigoClienteBuilder) {def codigo\_cliente\_builder}; %nó B
		\draw[<-] (Diretor) to (codigoClienteBuilder);
		\draw[<-] (BuilderConcreta) to[out=0,in=-80]  (codigoClienteBuilder);
		
	\end{tikzpicture}

\end{frame}




\begin{frame}[t]{Padrões de criação - Builder}
	

	\centering
	\begin{minipage}{4cm}
		\begin{block}{}
			Class ProdutoBase
		\end{block}
	\end{minipage} 
	$\underrightarrow{\makebox[2cm][r]{Possui}\hspace{2em}}$
	\begin{minipage}{6cm}
		\begin{block}{}
			Lógica do produto.
		\end{block}
	\end{minipage}
	\vfill
	\begin{minipage}{4cm}
		\begin{block}{}
			Class BuilderAbstrata
		\end{block}
	\end{minipage} 
	$\underrightarrow{\makebox[2cm][r]{Possui}\hspace{2em}}$
	\begin{minipage}{6cm}
		\begin{block}{}
			Interface para definir a construção do produto.
		\end{block}
	\end{minipage}
	\vfill
	\begin{minipage}{4cm}
		\begin{block}{}
			Class BuilderConcreta
		\end{block}
	\end{minipage} 
	$\underrightarrow{\makebox[2cm][r]{Possui}\hspace{2em}}$
	\begin{minipage}{6cm}
		\begin{block}{}
			Implementa a lógica para definir a construção do produto.
		\end{block}
	\end{minipage}
	\vfill
	\begin{minipage}{4cm}
		\begin{block}{}
			Class DiretorConcreta
		\end{block}
	\end{minipage} 
	$\underrightarrow{\makebox[2cm][r]{Possui}\hspace{2em}}$
	\begin{minipage}{6cm}
		\begin{block}{}
			Implementa a lógica para definir a ordem de construção do produto.
		\end{block}
	\end{minipage}
	\vfill
	\begin{minipage}{4cm}
		\begin{block}{}
			def codigo\_cliente\_builder
		\end{block}
	\end{minipage} 
	$\underrightarrow{\makebox[2cm][r]{Possui}\hspace{2em}}$
	\begin{minipage}{6cm}
		\begin{block}{}
			Implementa o código do cliente para obter o produto final.
		\end{block}
	\end{minipage}
\end{frame}





\begin{frame}[t]{Padrão de projeto - Builder}

	\lstinputlisting[style=CBruno,caption=Código da Classe Produto Base - codigo\_017\_produto\_base.py]{outros/codigos/python/exemplos-de-aulas/src/padroesdeprojetos/criacao/codigo_017_produto_base.py}

\end{frame}

\begin{frame}[t]{Padrão de projeto - Builder}

	\lstinputlisting[style=CBruno,caption=Código da Classe Builder Abtratato - codigo\_017\_builder\_abstrato.py]{outros/codigos/python/exemplos-de-aulas/src/padroesdeprojetos/criacao/codigo_017_builder_abstrato.py}
	
\end{frame}



\begin{frame}[t]{Padrão de projeto - Builder}

	\lstinputlisting[style=CBruno,caption=Código da Classe Builder Concreta - codigo\_017\_builder\_concreta.py]{outros/codigos/python/exemplos-de-aulas/src/padroesdeprojetos/criacao/codigo_017_builder_concreta.py}

\end{frame}


\begin{frame}[t]{Padrão de projeto - Builder}
	
	\lstinputlisting[style=CBruno,caption=Código da Classe Diretor Concreta - codigo\_017\_diretor\_base.py]{outros/codigos/python/exemplos-de-aulas/src/padroesdeprojetos/criacao/codigo_017_diretor_base.py}
	
\end{frame}


\begin{frame}[t]{Padrão de projeto - Builder}

	\lstinputlisting[style=CBruno,caption=Código da Função Código do cliente - codigo\_017\_codigo\_cliente\_builder.py]{outros/codigos/python/exemplos-de-aulas/src/padroesdeprojetos/criacao/codigo_017_codigo_cliente_builder.py}

\end{frame}



\begin{frame}[t]{Pytest}

	\lstinputlisting[style=CBruno,caption=Código do teste para o Código do cliente - test\_codigo\_017\_codigo\_cliente\_builder.py]{outros/codigos/python/exemplos-de-aulas/tests/test_codigo_017_codigo_cliente_builder.py}

\end{frame}




\begin{frame}[t]{Padrões de desenvolvimento de software}

	
	\fontsize{12pt}{15}\selectfont{
		\begin{itemize}%[<+->]  
			
			\item {\color{red}Padrões de Criação.}
			
			\begin{itemize}%[<+->]
				\item {\color{blue}Singleton \CheckmarkBold}
				\item {\color{blue}Abstract Factory \CheckmarkBold}
				\item {\color{blue}Factory Method \CheckmarkBold}
				\item {\color{blue}Builder \CheckmarkBold}
				\item Prototype \XSolidBrush
			\end{itemize}
			
		\end{itemize}
	}\par
	\vspace{1em}
	
	
\end{frame}





\begin{frame}[t]{Padrões de criação - Prototype}
	\fontsize{12pt}{15}\selectfont{
		{\color{blue}Prototype (Protótipo) \CheckmarkBold}
	}\par
	\vspace{1em}
	
	\fontsize{12pt}{15}\selectfont{
		\begin{itemize}%[<+->]
			
			\item Padrão criacional que permite a criação de novos objetos copiando ou clonando instâncias existentes, em vez de criar novas instâncias do zero.
			
			\item Isso é útil quando o custo de criação de um novo objeto é alto ou quando você deseja evitar a complexidade de inicializar um objeto em seu estado inicial.
			
		\end{itemize}
	}\par
	\vspace{1em}
\end{frame}



\begin{frame}[t]{Padrões de criação - Prototype}
	\fontsize{12pt}{15}\selectfont{
		Exemplo
	}\par
	\vspace{0.5em}
	
	\fontsize{12pt}{15}\selectfont{
		\begin{enumerate}%[<+->]
			
			\item \textbf{Classe PrototypeBase} - Uma interface ou classe abstrata que define o método clone(), que é responsável por criar uma cópia do objeto.
			
			\item \textbf{Classe PrototypeConcreta1 e PrototypeConcreta2} - Classes concretas que implementam a interface Prototype e sobrescrevem o método clone() para retornar uma cópia do objeto.
			
			\item \textbf{Código do cliente para utilizar o Padrão Prototype} - O local que utiliza o método clone() para criar novos objetos a partir de protótipos.
			
		\end{enumerate}
	}\par
	\vspace{1em}
\end{frame}




\begin{frame}[t]{Padrões de criação - Prototype}
	\fontsize{14pt}{15}\selectfont{
		Resumo
	}\par
	\vspace{1em}
	
	\fontsize{12pt}{15}\selectfont{
		\begin{itemize}%[<+->]
			
			\item Prototype:  Define a interface clone() que as subclasses precisam implementar.
			
			\item PrototypeConcreta1 e PrototypeConcreta2: Implementam a interface Prototype e fornecem a implementação do método clone() que usa copy.deepcopy() para criar uma cópia profunda do objeto.
			
			\item Cliente: O cliente cria novos objetos clonando os protótipos.
			
		\end{itemize}
	}\par
	\vspace{1em}
\end{frame}



\begin{frame}[t]{Padrões de criação - Prototype}
	\fontsize{16pt}{15}\selectfont{
		Aplicação
	}\par
	\vspace{1em}
	
	\fontsize{14pt}{15}\selectfont{
		\begin{itemize}%[<+->]
			
			\item Quando o custo de criação de um novo objeto é muito caro ou complexo.
			
			\item Quando você deseja evitar a duplicação do estado de configuração dos objetos.
			
			\item Quando você precisa de uma variedade de objetos semelhantes.
			
		\end{itemize}
	}\par
	\vspace{1em}
	
	O padrão Prototype é particularmente útil em cenários onde os objetos têm um estado inicial complexo ou quando você precisa criar cópias de objetos com uma configuração específica que não pode ser facilmente reproduzida.
\end{frame}




\begin{frame}[t]{Padrões de criação - Prototype}
	
	\vspace{1em}
	\begin{tikzpicture}[
		every node/.style={
			draw=blue!50!white,
			minimum width=2.5in,
			minimum height=1.5em,
			font=\sffamily,
			anchor=north,
		},
		connect/.style={
			draw=blue!70!white,
			-stealth,
		},
		]
		
		\node[draw] (PrototypeBase) at (0,0) {Class PrototypeBase}; %nó A
		\node[below=of PrototypeBase] (PrototypeConcreta1) {Class PrototypeConcreta1}; %nó B
		\draw[<-] (PrototypeBase) to (PrototypeConcreta1);
		
		\node[below right=of PrototypeConcreta1] (PrototypeConcreta2) {Class PrototypeConcreta2}; %nó B
		\draw[<-] (PrototypeBase) to (PrototypeConcreta2);
		
		\node[right=of PrototypeBase] (Cliente) {Cliente}; %nó B
		\draw[<-] (PrototypeBase) to (Cliente);

	\end{tikzpicture}
	
\end{frame}




\begin{frame}[t]{Padrões de criação - Prototype}
	
	
	\centering
	\begin{minipage}{4cm}
		\begin{block}{}
			Class PrototypeBase
		\end{block}
	\end{minipage} 
	$\underrightarrow{\makebox[2cm][r]{Possui}\hspace{2em}}$
	\begin{minipage}{6cm}
		\begin{block}{}
			Interface para definir o clone de objeto.
		\end{block}
	\end{minipage}
	\vfill
	\begin{minipage}{4cm}
		\begin{block}{}
			Class PrototypeConcreta1
		\end{block}
	\end{minipage} 
	$\underrightarrow{\makebox[2cm][r]{Possui}\hspace{2em}}$
	\begin{minipage}{6cm}
		\begin{block}{}
			Implementa a lógica para definir o clone do objeto PrototypeConcreta1.
		\end{block}
	\end{minipage}
	\vfill
	\begin{minipage}{4cm}
		\begin{block}{}
			Class PrototypeConcreta2
		\end{block}
	\end{minipage} 
	$\underrightarrow{\makebox[2cm][r]{Possui}\hspace{2em}}$
	\begin{minipage}{6cm}
		\begin{block}{}
			Implementa a lógica para definir o clone do objeto PrototypeConcreta2.
		\end{block}
	\end{minipage}
	\vfill
	\begin{minipage}{4cm}
		\begin{block}{}
			def codigo\_cliente\_prototype
		\end{block}
	\end{minipage} 
	$\underrightarrow{\makebox[2cm][r]{Possui}\hspace{2em}}$
	\begin{minipage}{6cm}
		\begin{block}{}
			Implementa o código do cliente para obter o clone.
		\end{block}
	\end{minipage}
\end{frame}





\begin{frame}[t]{Padrão de projeto - Prototype}
	
	\lstinputlisting[style=CBruno,caption=Código da Classe Prototype Base - codigo\_018\_prototype\_base.py]{outros/codigos/python/exemplos-de-aulas/src/padroesdeprojetos/criacao/codigo_018_prototype_base.py}
	
\end{frame}


\begin{frame}[t]{Padrão de projeto - Prototype}
	
	\lstinputlisting[style=CBruno,caption=Código da Classe Prototype Concreta 1 - codigo\_018\_prototype\_concreta1.py]{outros/codigos/python/exemplos-de-aulas/src/padroesdeprojetos/criacao/codigo_018_prototype_concreta1.py}
	
	\vspace{1cm}
	\url{https://docs.python.org/pt-br/3/library/copy.html}
	
\end{frame}


\begin{frame}[t]{Padrão de projeto - Prototype}
	
	\lstinputlisting[style=CBruno,caption=Código da Classe Prototype Concreta 2 - codigo\_018\_prototype\_concreta2.py]{outros/codigos/python/exemplos-de-aulas/src/padroesdeprojetos/criacao/codigo_018_prototype_concreta2.py}
	
	\vspace{1cm}
	\url{https://docs.python.org/pt-br/3/library/copy.html}
	
\end{frame}


\begin{frame}[t]{Padrão de projeto - Prototype}
	
	\lstinputlisting[style=CBruno,caption=Código da Função Código do cliente - codigo\_018\_codigo\_cliente\_prototype.py]{outros/codigos/python/exemplos-de-aulas/src/padroesdeprojetos/criacao/codigo_018_codigo_cliente_prototype.py}
	
\end{frame}


\begin{frame}[t]{Pytest}
	
	\lstinputlisting[style=CBruno,caption=Código do teste para o Código do cliente - test\_codigo\_018\_codigo\_cliente\_prototype.py]{outros/codigos/python/exemplos-de-aulas/tests/test_codigo_018_codigo_cliente_prototype.py}
	
\end{frame}


\pgfplotstablegetelem{\thepart}{[index]\columnIndex}\of{\cronograma}
\part{\pgfplotsretval}
\label{part:\thepart}
\frame{\partpage}


\begin{frame}[t]{Padrões de Desenvolvimento de Software}
	
	\fontsize{12pt}{15.2}\selectfont{
		
		...São soluções típicas para problemas comuns em projeto de software.
		
	}\par
	\vspace{1em}
	
	
	\fontsize{12pt}{15}\selectfont{
		\begin{itemize}%[<+->]  
			
			\item {\color{red}Reusabilidade de Software.}
			\item {\color{red}Conceitos e uso de SOLID.}
			\item Conceitos básicos de Padrões de Projeto.
			\item Padrões de Criação.
			\item Padrões Estruturais.
			\item Padrões Comportamentais.
			\item Padrões Arquiteturais(MVC).
			
			
		\end{itemize}
	}\par
	\vspace{1em}
	
	
\end{frame}


\begin{frame}[t]{Padrões de Desenvolvimento de Software}
	
	\fontsize{14pt}{15.2}\selectfont{
		
		Reúso de software é o processo de incorporar produtos existentes em um novo produto.
		
	}\par
	\vspace{1em}
	
	
	\fontsize{12pt}{15}\selectfont{
		\begin{itemize}%[<+->]  
			
			\item Código.
			\item specificações de Requisitos e Projeto.
			\item Planos de Teste.
			\item Conhecimento.
						
		\end{itemize}
	}\par
	\vspace{1em}
	
	
\end{frame}




\begin{frame}[t]{Padrões de Desenvolvimento de Software}
	
	\fontsize{14pt}{15.2}\selectfont{
		Benefícios
		
	}\par
	\vspace{1em}
	
	\fontsize{12pt}{15}\selectfont{
		\begin{itemize}%[<+->]  
			
			\item Aumento da Produtividade.
			\item Diminuição do tempo de desenvolvimento e validação -> Redução	de custo.
			\item Qualidade dos Produtos.
			\item Flexibilidade na estrutura do software.
			\item Manutenibilidade.
			\item Familiaridade com o uso de padrões -> leva a menos erros.
			
		\end{itemize}
	}\par
	\vspace{1em}
	
\end{frame}



\begin{frame}[t]{Padrões de Desenvolvimento de Software}
	
	\fontsize{14pt}{15.2}\selectfont{
		Dificuldades
		
	}\par
	\vspace{1em}
	
	\fontsize{12pt}{15}\selectfont{
		\begin{itemize}%[<+->]  
			
			\item Identificação e compreensão dos artefatos.
			\item Qualidade dos artefatos.
			\item Modificação dos artefatos.
			\item Falta de confiança nos “artefatos dos outros”. Mito: “não inventado aqui.”.
			\item Ferramentas de apoio.
			\item Aspectos legais e econômicos.
			\item Falta de incentivo.
			
		\end{itemize}
	}\par
	\vspace{1em}
	
\end{frame}




\begin{frame}[t]{Padrões de Desenvolvimento de Software}
	
	\fontsize{14pt}{15.2}\selectfont{
		Requisitos
		
	}\par
	\vspace{1em}
	
	\fontsize{12pt}{15}\selectfont{
		\begin{itemize}%[<+->]  
			
			\item Catalogação, documentação e certificação completa do artefato a
			ser reutilizado, de modo a ser possível:
			 
			\begin{itemize}%[<+->]		
				\item Encontrar o artefato a ser reutilizado.
				\item Compreender o artefato para adaptá-lo ao novo contexto.
				\item Garantir que o artefato se comportará conforme especificado.
				
			\end{itemize}
			
		\end{itemize}
	}\par
	\vspace{1em}
	
\end{frame}



\begin{frame}[t]{Padrões de Desenvolvimento de Software}
	
	\fontsize{14pt}{15.2}\selectfont{
		Uma boa técnica de reúso deve garantir adaptação e adequação a um novo contexto:
		
	}\par
	\vspace{1em}
	
	\fontsize{12pt}{15}\selectfont{
		\begin{itemize}%[<+->]  
			
			\item Abstração.
			\item Seleção.
			\item Especialização.
			\item Integração.
			
		\end{itemize}
	}\par
	\vspace{1em}
	
\end{frame}



\begin{frame}[t]{Padrões de Desenvolvimento de Software}
	
	\fontsize{14pt}{15.2}\selectfont{
		Técnicas para Reúso.
		
	}\par
	\vspace{1em}
	
	\fontsize{12pt}{15}\selectfont{
		\begin{itemize}%[<+->]  
			
			\item Bibliotecas.
			\item Frameworks.
			\item Componentes.
			\item Padrões de Software.
			\item Linhas de Produto de Software.
			
		\end{itemize}
	}\par
	\vspace{1em}
	
\end{frame}






\begin{frame}[t]{Padrões de Desenvolvimento de Software}
	
	\vspace{7em}
	\centering
	\fontsize{46pt}{15.2}\selectfont{
		
		Princípios de projetos S.O.L.I.D.
		
	}\par
	\vspace{1em}
		
	
\end{frame}


\begin{frame}[t]{Padrões de Desenvolvimento de Software}
	
	\fontsize{12pt}{15.2}\selectfont{
		
		SOLID - são cinco princípios de design de \textbf{código orientado a objeto} que basicamente tem os seguintes objetivos.
		
	}\par
	\vspace{1em}
	
	
	\fontsize{12pt}{15}\selectfont{
		\begin{itemize}%[<+->]  
			
			\item Tornar o código mais entendível, claro e conciso.
			\item Tornar o código mais flexível e tolerante a mudanças.
			\item Aumentar a adesão do código aos princípios da orientação a objetos.
			
			
		\end{itemize}
	}\par
	\vspace{1em}
	
	
\end{frame}



\begin{frame}[t]{Padrões de Desenvolvimento de Software}
	
	\fontsize{12pt}{15.2}\selectfont{
		
		SOLID é um acrônimo para cada um dos cinco princípios que fazem parte desse grupo:
		
	}\par
	\vspace{1em}
	
	
	\fontsize{12pt}{15}\selectfont{
		\begin{itemize}%[<+->]  
			
			\item \textit{Single Responsability Principle} (Princípio da Responsabilidade Única).
			\item \textit{Open/Closed Principle} (Princípio do “Aberto para Extensão/Fechado para Implementação).
			\item \textit{Liskov Substitution Principle} (Princípio da Substituição de Liskov).
			\item \textit{Interface Segregation Principle} (Princípio da Segregação de Interfaces).
			\item \textit{Dependency Inversion Principle} (Princípio da Inversão de Dependências).
			
			
			
		\end{itemize}
	}\par
	\vspace{1em}
	
	
\end{frame}



\begin{frame}[t]{Padrões de Desenvolvimento de Software}
	
	\fontsize{14pt}{15.2}\selectfont{		
		\textit{Single Responsability Principle} (Princípio da Responsabilidade Única).
	}\par
	\vspace{1em}	
	
	\fontsize{12pt}{15}\selectfont{
		\begin{itemize}%[<+->]  
			
			\item Uma classe deve ter UM, e somente um, MOTIVO para mudar.
			\item Alterações causam “incerteza”
			\begin{itemize}%[<+->]  
				\item Cada linha modificada pode introduzir BUG novo.
				\item Diminui a coesão e aumenta acoplamento.
			\end{itemize}
			\item É o padrão “base”.
			\item Anti-padrões: Classe Deus/Grande bola de lama/etc.
		
		\end{itemize}
	}\par
	\vspace{1em}
	
\end{frame}



\begin{frame}[t]{Padrões de Desenvolvimento de Software}
	
	\fontsize{14pt}{15.2}\selectfont{		
		\textit{Open/Closed Principle} (Princípio do “Aberto para Extensão/Fechado para Implementação).
	}\par
	\vspace{1em}	
	
	\fontsize{11pt}{13}\selectfont{
		\begin{itemize}%[<+->]  
			
			\item Objetos e/ou classes devem estar abertos para extensão, mas
			fechados para modificação.
			\begin{itemize}%[<+->]  
				\item É possível incluir novas funcionalidades.
			\end{itemize}
			\item Alterações causam “incerteza” (de novo).
			\item Abstração é a chave
			\item Utilização de bom encapsulamento.
			\begin{itemize}%[<+->]  
				\item Atributos sempre privados.
				\item Uso de polimorfismo: criação de interfaces/classes abstratas.
				\item Jamais usar variáveis globais ou “similares”.
			\end{itemize}
			\item Anti-padrões: código espaguete.
			
		\end{itemize}
	}\par
	\vspace{1em}
	
\end{frame}



\begin{frame}[t]{Padrões de Desenvolvimento de Software}
	
	\fontsize{14pt}{15.2}\selectfont{		
		\textit{Liskov Substitution Principle} (Princípio da Substituição de Liskov).
	}\par
	\vspace{1em}	
	
	\fontsize{11pt}{13}\selectfont{
		\begin{itemize}%[<+->]  
			
			\item Uma classe derivada deve ser substituível pela sua classe base.
			\begin{itemize}%[<+->]  
				\item “Se para cada objeto o1 do tipo S há um objeto o2 do tipo T de forma que, para todos os
				programas P definidos em termos de T, o comportamento de P é inalterado quando o1 é
				substituído por o2 então S é um subtipo de T.
			\end{itemize}
			\item Utilização de poliformismo adequado.
			\begin{itemize}%[<+->]  
				\item A validade do modelo depende de seus filhos.
				\item Relacionamento IS-A ligado ao comportamento.
				\begin{itemize}%[<+->]  
					\item Problemas em CASTs.
				\end{itemize}
			\end{itemize}
			\item Pode ser relacionado com “Design por contrato”.
			\begin{itemize}%[<+->]  
				\item Pré condições e pós condições na execução.	
			\end{itemize}
			
		\end{itemize}
	}\par
	\vspace{1em}
	
\end{frame}




\begin{frame}[t]{Padrões de Desenvolvimento de Software}
	
	\fontsize{14pt}{15.2}\selectfont{		
		\textit{Interface Segregation Principle} (Princípio da Segregação de Interfaces).
	}\par
	\vspace{1em}	
	
	\fontsize{13pt}{15}\selectfont{
		\begin{itemize}%[<+->]  
			
			\item Classe não implementa interface com métodos que não vai usar.
			\item Evitar poluição da interface -> baixo acoplamento.
			\item Anti-padrão: interface “Deus”/martelo de ouro.
			
		\end{itemize}
	}\par
	\vspace{1em}
	
\end{frame}



\begin{frame}[t]{Padrões de Desenvolvimento de Software}
	
	\fontsize{14pt}{15.2}\selectfont{		
		\textit{Dependency Inversion Principle} (Princípio da Inversão de Dependências).
	}\par
	\vspace{1em}	
	
	\fontsize{13pt}{15}\selectfont{
		\begin{itemize}%[<+->]  
			
			\item Módulos de alto nível não devem depender de módulos de baixo nível.
			Ambos devem depender de abstrações.
			\item Abstrações não devem depender de detalhes. Detalhes (implementações)
			devem depender de abstrações.
			\item Utilização constante de polimorfismo.
				\begin{itemize}%[<+->]  
					\item Facilita a reutilização.
				\end{itemize}
			\item Anti-padrão: complexidade “acidental”/gambiarra/copiar-colar.
		\end{itemize}
	}\par
	\vspace{1em}
	
\end{frame}



\pgfplotstablegetelem{\thepart}{[index]\columnIndex}\of{\cronograma}
\part{\pgfplotsretval}
\label{part:\thepart}
\frame{\partpage}




\begin{frame}[t]{Padrões de desenvolvimento de software}
	
	
	\fontsize{12pt}{15}\selectfont{
		\begin{itemize}%[<+->]  
			
			\item {\color{red}Padrões de Criação.}
			
			\begin{itemize}%[<+->]
				\item {\color{blue}Singleton \CheckmarkBold}
				\item {\color{blue}Abstract Factory \CheckmarkBold}
				\item {\color{blue}Factory Method \CheckmarkBold}
				\item {\color{blue}Builder \CheckmarkBold}
				\item Prototype \XSolidBrush
			\end{itemize}
			
		\end{itemize}
	}\par
	\vspace{1em}
	
	
\end{frame}





\begin{frame}[t]{Padrões de criação - Prototype}
	\fontsize{12pt}{15}\selectfont{
		{\color{blue}Prototype (Protótipo) \CheckmarkBold}
	}\par
	\vspace{1em}
	
	\fontsize{12pt}{15}\selectfont{
		\begin{itemize}%[<+->]
			
			\item Padrão criacional que permite a criação de novos objetos copiando ou clonando instâncias existentes, em vez de criar novas instâncias do zero.
			
			\item Isso é útil quando o custo de criação de um novo objeto é alto ou quando você deseja evitar a complexidade de inicializar um objeto em seu estado inicial.
			
		\end{itemize}
	}\par
	\vspace{1em}
\end{frame}



\begin{frame}[t]{Padrões de criação - Prototype}
	\fontsize{12pt}{15}\selectfont{
		Exemplo
	}\par
	\vspace{0.5em}
	
	\fontsize{12pt}{15}\selectfont{
		\begin{enumerate}%[<+->]
			
			\item \textbf{Classe PrototypeBase} - Uma interface ou classe abstrata que define o método clone(), que é responsável por criar uma cópia do objeto.
			
			\item \textbf{Classe PrototypeConcreta1 e PrototypeConcreta2} - Classes concretas que implementam a interface Prototype e sobrescrevem o método clone() para retornar uma cópia do objeto.
			
			\item \textbf{Código do cliente para utilizar o Padrão Prototype} - O local que utiliza o método clone() para criar novos objetos a partir de protótipos.
			
		\end{enumerate}
	}\par
	\vspace{1em}
\end{frame}




\begin{frame}[t]{Padrões de criação - Prototype}
	\fontsize{14pt}{15}\selectfont{
		Resumo
	}\par
	\vspace{1em}
	
	\fontsize{12pt}{15}\selectfont{
		\begin{itemize}%[<+->]
			
			\item Prototype:  Define a interface clone() que as subclasses precisam implementar.
			
			\item PrototypeConcreta1 e PrototypeConcreta2: Implementam a interface Prototype e fornecem a implementação do método clone() que usa copy.deepcopy() para criar uma cópia profunda do objeto.
			
			\item Cliente: O cliente cria novos objetos clonando os protótipos.
			
		\end{itemize}
	}\par
	\vspace{1em}
\end{frame}



\begin{frame}[t]{Padrões de criação - Prototype}
	\fontsize{16pt}{15}\selectfont{
		Aplicação
	}\par
	\vspace{1em}
	
	\fontsize{14pt}{15}\selectfont{
		\begin{itemize}%[<+->]
			
			\item Quando o custo de criação de um novo objeto é muito caro ou complexo.
			
			\item Quando você deseja evitar a duplicação do estado de configuração dos objetos.
			
			\item Quando você precisa de uma variedade de objetos semelhantes.
			
		\end{itemize}
	}\par
	\vspace{1em}
	
	O padrão Prototype é particularmente útil em cenários onde os objetos têm um estado inicial complexo ou quando você precisa criar cópias de objetos com uma configuração específica que não pode ser facilmente reproduzida.
\end{frame}




\begin{frame}[t]{Padrões de criação - Prototype}
	
	\vspace{1em}
	\begin{tikzpicture}[
		every node/.style={
			draw=blue!50!white,
			minimum width=2.5in,
			minimum height=1.5em,
			font=\sffamily,
			anchor=north,
		},
		connect/.style={
			draw=blue!70!white,
			-stealth,
		},
		]
		
		\node[draw] (PrototypeBase) at (0,0) {Class PrototypeBase}; %nó A
		\node[below=of PrototypeBase] (PrototypeConcreta1) {Class PrototypeConcreta1}; %nó B
		\draw[<-] (PrototypeBase) to (PrototypeConcreta1);
		
		\node[below right=of PrototypeConcreta1] (PrototypeConcreta2) {Class PrototypeConcreta2}; %nó B
		\draw[<-] (PrototypeBase) to (PrototypeConcreta2);
		
		\node[right=of PrototypeBase] (Cliente) {Cliente}; %nó B
		\draw[<-] (PrototypeBase) to (Cliente);
		
	\end{tikzpicture}
	
\end{frame}




\begin{frame}[t]{Padrões de criação - Prototype}
	
	
	\centering
	\begin{minipage}{4cm}
		\begin{block}{}
			Class PrototypeBase
		\end{block}
	\end{minipage} 
	$\underrightarrow{\makebox[2cm][r]{Possui}\hspace{2em}}$
	\begin{minipage}{6cm}
		\begin{block}{}
			Interface para definir o clone de objeto.
		\end{block}
	\end{minipage}
	\vfill
	\begin{minipage}{4cm}
		\begin{block}{}
			Class PrototypeConcreta1
		\end{block}
	\end{minipage} 
	$\underrightarrow{\makebox[2cm][r]{Possui}\hspace{2em}}$
	\begin{minipage}{6cm}
		\begin{block}{}
			Implementa a lógica para definir o clone do objeto PrototypeConcreta1.
		\end{block}
	\end{minipage}
	\vfill
	\begin{minipage}{4cm}
		\begin{block}{}
			Class PrototypeConcreta2
		\end{block}
	\end{minipage} 
	$\underrightarrow{\makebox[2cm][r]{Possui}\hspace{2em}}$
	\begin{minipage}{6cm}
		\begin{block}{}
			Implementa a lógica para definir o clone do objeto PrototypeConcreta2.
		\end{block}
	\end{minipage}
	\vfill
	\begin{minipage}{4cm}
		\begin{block}{}
			def codigo\_cliente\_prototype
		\end{block}
	\end{minipage} 
	$\underrightarrow{\makebox[2cm][r]{Possui}\hspace{2em}}$
	\begin{minipage}{6cm}
		\begin{block}{}
			Implementa o código do cliente para obter o clone.
		\end{block}
	\end{minipage}
\end{frame}





\begin{frame}[t]{Padrão de projeto - Prototype}
	
	%	\lstinputlisting[style=CBruno,caption=Código da Classe Prototype Base - codigo\_018\_prototype\_base.py]{}
	
	\centering
	\makebox[\linewidth][c]{
		\begin{minipage}{0.95\textwidth}
			\inputminted[baselinestretch=1.25,fontsize={\fontsize{13}{10}\selectfont}]{python}{outros/codigos/python/exemplos-de-aulas/src/padroesdeprojetos/criacao/codigo_018_prototype_base.py}
		\end{minipage}
	}
	\fontsize{7pt}{6}\selectfont{
		Código da Classe Prototype Base - codigo\_018\_prototype\_base.py
	}\par
	
	
\end{frame}


\begin{frame}[t]{Padrão de projeto - Prototype}
	
	%	\lstinputlisting[style=CBruno,caption=Código da Classe Prototype Concreta 1 - codigo\_018\_prototype\_concreta1.py]{outros/codigos/python/exemplos-de-aulas/src/padroesdeprojetos/criacao/codigo_018_prototype_concreta1.py}
	
	\centering
	\makebox[\linewidth][c]{
		\begin{minipage}{0.95\textwidth}
			\inputminted[fontsize={\fontsize{10}{8}\selectfont}]{python}{outros/codigos/python/exemplos-de-aulas/src/padroesdeprojetos/criacao/codigo_018_prototype_concreta1.py}
		\end{minipage}
	}
	\fontsize{7pt}{6}\selectfont{
		Código da Classe Prototype Concreta 1 - codigo\_018\_prototype\_concreta1.py
	}\par
	
	\vspace{1cm}
	\url{https://docs.python.org/pt-br/3/library/copy.html}
	
\end{frame}


\begin{frame}[t]{Padrão de projeto - Prototype}
	
	%	\lstinputlisting[style=CBruno,caption=Código da Classe Prototype Concreta 2 - codigo\_018\_prototype\_concreta2.py]{outros/codigos/python/exemplos-de-aulas/src/padroesdeprojetos/criacao/codigo_018_prototype_concreta2.py}
	
	
	\centering
	\makebox[\linewidth][c]{
		\begin{minipage}{0.95\textwidth}
			\inputminted[fontsize={\fontsize{10}{8}\selectfont}]{python}{outros/codigos/python/exemplos-de-aulas/src/padroesdeprojetos/criacao/codigo_018_prototype_concreta2.py}
		\end{minipage}
	}
	\fontsize{7pt}{6}\selectfont{
		Código da Classe Prototype Concreta 2 - codigo\_018\_prototype\_concreta2.py
	}\par
	
	\vspace{0.5cm}
	\url{https://docs.python.org/pt-br/3/library/copy.html}
	
\end{frame}


\begin{frame}[t]{Padrão de projeto - Prototype}
	
	%	\lstinputlisting[style=CBruno,caption=Código da Função Código do cliente - codigo\_018\_codigo\_cliente\_prototype.py]{outros/codigos/python/exemplos-de-aulas/src/padroesdeprojetos/criacao/codigo_018_codigo_cliente_prototype.py}
	
	\begin{listing}[H]
		\centering
		\makebox[\linewidth][c]{
			\begin{minipage}{0.95\textwidth}
				\inputminted[baselinestretch=1,fontsize={\fontsize{7}{6}\selectfont}]{python}{outros/codigos/python/exemplos-de-aulas/src/padroesdeprojetos/criacao/codigo_018_codigo_cliente_prototype.py}
			\end{minipage}
		}
		\fontsize{7pt}{6}\selectfont{
			Código da Função Código do cliente - codigo\_018\_codigo\_cliente\_prototype.py
		}\par
	\end{listing}
	
\end{frame}


\begin{frame}[t]{Testes Unitários}
	
	%	\lstinputlisting[style=CBruno,caption=Código do teste para o Código do cliente - test\_codigo\_018\_codigo\_cliente\_prototype.py]{outros/codigos/python/exemplos-de-aulas/tests/test_codigo_018_codigo_cliente_prototype.py}
	%
	
	\begin{listing}[H]
		\centering
		\makebox[\linewidth][c]{
			\begin{minipage}{0.95\textwidth}
				\inputminted[baselinestretch=1.25,fontsize={\fontsize{10}{8}\selectfont}]{python}{outros/codigos/python/exemplos-de-aulas/tests/test_codigo_018_codigo_cliente_prototype.py}
			\end{minipage}
		}
		\fontsize{7pt}{6}\selectfont{
			Código do teste para o Código do cliente - test\_codigo\_018\_codigo\_cliente\_prototype.py
		}\par
	\end{listing}
	
	
\end{frame}






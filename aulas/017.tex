\pgfplotstablegetelem{\thepart}{[index]\columnIndex}\of{\cronograma}
\part{\pgfplotsretval}
\label{part:\thepart}
\frame{\partpage}


\begin{frame}[t]{Programação Orientada a Objetos}
	
	\fontsize{14pt}{15}\selectfont{
		
		...continuando a partir da aula passada sobre \gls{poo}.
		
	}\par
	\vspace{1em}
	
	
\end{frame}




\begin{frame}[t]{Programação Orientada a Objetos}
	
	\fontsize{14pt}{15}\selectfont{
		
		...relembrando a última aula:
				
	}\par
	\vspace{1em}
	
	\fontsize{12pt}{15}\selectfont{
		\begin{itemize}%[<+->] 
			
			\item Na terminologia Python a classe original é chamada de Classe Base, Classe Mãe ou Superclasse (Base, Parent ou Super Class) enquanto que a nova é chamada de Sub Classe ou Classe Filha (Sub ou Child Class).
			
			\item A subclasse pode especializar a classe principal ou mesmo estendê-la com novos métodos e atributos.
			
		\end{itemize}
	}\par
	\vspace{1em}
	
\end{frame}




\begin{frame}[t]{Herança em classes}

	\centering
	\makebox[\linewidth][c]{
		\begin{minipage}{0.95\textwidth}
			\inputminted[baselinestretch=1.25,fontsize={\fontsize{8}{9}\selectfont}]{python}{outros/codigos/python/exemplos-de-aulas/src/codigo_019_classe_produto_e_produto_critico.py}
		\end{minipage}
	}
	\fontsize{7pt}{6}\selectfont{
		Código das Classes Produto, ProdutoCrítico e Código Cliente - codigo\_019\_classe\_produto\_e\_produto\_critico.py
	}\par

\end{frame}


\begin{frame}[t]{Teste Unitário}
	
	\centering
	\makebox[\linewidth][c]{
		\begin{minipage}{0.95\textwidth}
			\inputminted[baselinestretch=1.25,fontsize={\fontsize{9}{9}\selectfont}]{python}{outros/codigos/python/exemplos-de-aulas/tests/test_codigo_019_classe_produto_e_produto_critico.py}
		\end{minipage}
	}
	\fontsize{7pt}{6}\selectfont{
		Código das Classes Produto, ProdutoCrítico e Código do Cliente - test\_codigo\_019\_classe\_produto\_e\_produto\_critico.py
	}\par
	
\end{frame}


\begin{frame}[t]{Exercícios}
	\fontsize{10pt}{15}\selectfont{
		Escreva o algoritmo e programa. Use Herança de \gls{poo} para resolver.
		\begin{itemize}%[<+->]
			\item \glsfirst{exercicio_018}: \glsdesc{exercicio_018}
			\vspace{0.5em}
%			\includegraphics[scale=0.06]{imagens/fig-atencao-pessoas-estudando.png}
		\end{itemize}
	}\par
	\vspace{1em}
\end{frame}


\begin{frame}[t]{Algoritmo}
	\fontsize{9pt}{9}\selectfont{
	\begin{itemize}%[<+->]
		\item Ler quantidade atual (qtd\_atual) quantidade máxima (qtd\_max) e quantidade mínima em estoque (qtd\_min).
		\item Obtenha a quantidade média do estoque, dado por qtd\_media = (qtd\_max + qtd\_min)/2.
		\item Obtenha a quantidade média do estoque crítico, dado por qtd\_media\_critico = (qtd\_max - qtd\_min)/2.
		\item Obtenha a situação do estoque, dada por\\ 
		situacao\_estoque = Se qtd\_atual >= qtd\_media, então 'Não efetuar compra' SeNão 'Efetuar compra'
		\item Obtenha a situação do estoque crítico, dada por\\ 
		situacao\_estoque\_critico = Se qtd\_atual >= qtd\_media\_critico, então 'Não efetuar compra' SeNão 'Efetuar compra'
		\item Escreva a quantidade média do estoque qtd\_media.
		\item Escreva a quantidade média do estoque crítico qtd\_media\_critico.
		\item Escreva a situação do estoque
		\item Escreva a situação do estoque crítico
	\end{itemize}
	}\par
	\vspace{0.5em}
	Exemplo:\\
	entradas de dados 220,400 e 50.\\
	Saídas\\
	quantidade média do estoque: 225\\
	quantidade média do estoque: 175\\
	situação do estoque: Efetuar compra \# 400+50=450, 450/2=225, 220>=225, SeNão... \\
	situação do estoque crítico: Não efetuar compra \# 400-50=350, 350/2=175, 220>=175, então...
	
\end{frame}









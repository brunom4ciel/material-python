\pgfplotstablegetelem{\thepart}{[index]\columnIndex}\of{\cronograma}
\part{\pgfplotsretval}
\label{part:\thepart}
\frame{\partpage}


\begin{frame}[t]{Padrões de desenvolvimento de software}
	\fontsize{14pt}{15.2}\selectfont{
		Design patterns, também conhecidos como padrões de design ou padrões de projeto, são soluções já testadas e aprovadas em problemas recorrentes durante o desenvolvimento de software.
		
	}\par
	\vspace{1em}
	
	\fontsize{10pt}{15}\selectfont{
		\begin{itemize}%[<+->]  
			\item  \textbf{abstração}: os padrões representam um conhecimento aplicado no cotidiano.
			
			\item \textbf{encapsulamento}: um problema é retratado como uma cápsula, que deve ser independente, específica e objetiva. O mesmo vale para uma solução já definida.
			
			\item \textbf{combinatoriedade}: há uma hierarquia entre os padrões, do mais alto ao mais baixo.
			
			\item \textbf{equilíbrio}: cada passo dentro de um projeto de software precisa buscar o equilíbrio.
			
			\item \textbf{abertura}: os padrões precisam permitir uma extensão até os níveis mais baixos.
			
			\item \textbf{generalidade}: todo padrão deve servir de base para a construção de outros projetos.
			
		\end{itemize}
	}\par
	\vspace{1em}
	
\end{frame}


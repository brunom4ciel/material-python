\pgfplotstablegetelem{\thepart}{[index]\columnIndex}\of{\cronograma}
\part{\pgfplotsretval}
\label{part:\thepart}
\frame{\partpage}


\begin{frame}[t]{Exemplos de criação de classe}
	
	\fontsize{14pt}{15}\selectfont{
		
		...continuando a partir da aula passada sobre \gls{poo}.
		
	}\par
	\vspace{1em}
	
	
\end{frame}




\begin{frame}[t]{Tratamento de exceções}
	
	\fontsize{13pt}{15}\selectfont{
		
		O tratamento de exceção, na ciência da computação, é o mecanismo responsável pelo tratamento da ocorrência de condições que alteram o fluxo normal da execução de programas de computadores. \\
		\vspace{0.5em}
		Para condições consideradas parte do fluxo normal de execução, ver os conceitos de sinal e evento.
		
	}\par
	
	\vspace{2em}
	Saiba mais em\\
	\fontsize{10pt}{15}\selectfont{
		\url{https://docs.python.org/pt-br/3/whatsnew/2.6.html\#pep-3110-exception-handling-changes}\\
		\url{https://docs.python.org/pt-br/3/reference/compound_stmts.html\#except-clause}\\
		\url{https://docs.pytest.org/en/stable/how-to/assert.html}
	}\par
\end{frame}


\begin{frame}[t]{Tratamento de exceções}
	
	\vspace{1em}
	
	\fontsize{13pt}{15}\selectfont{
		A sintaxe básica é:
	}\par
	
	\vspace{1em}
	\begin{beamercolorbox}[wd=\textwidth]{warning}
		try: \\
		\hspace{1em}{Instruções} \# o código da funcionalidade.\\
		...\\
		except <ExceptType>:\\
		\hspace{1em}{Instruções} \# o código para tratamento da exceção.\\
		...\\
		finally: \# Caso o fluxo não seja interrompido, sempre é executado o finally.\\
		\hspace{1em}{Instruções}
	\end{beamercolorbox}
	
	
	\vspace{2em}
	\url{https://docs.python.org/pt-br/3/tutorial/errors.html}
	
\end{frame}


\begin{frame}[t]{Tratamento de exceções}
	
	\lstinputlisting[style=CBruno,caption=Tratamento de exceções]{outros/codigos/python/exemplos-de-aulas/src/codigo_006_classe_veiculo_com_exception.py}
	
	
\end{frame}


\begin{frame}[t]{Pytest}
	
	%	\vspace{-2em}
	\lstinputlisting[style=CBruno,caption=Cobertura de testes da classe Veiculo]{outros/codigos/python/exemplos-de-aulas/tests/test_codigo_006_classe_veiculo_com_exception.py}
	
	
\end{frame}














\begin{frame}[t]{Polimorfismo em classe}
	
	\fontsize{14pt}{15}\selectfont{
		
		Polimorfismo permite que objetos de diferentes classes sejam tratados como objetos de uma classe comum.
		
	}\par
	\vspace{1em}
	
	\fontsize{12pt}{15}\selectfont{
		\begin{itemize}%[<+->] 
			
			\item Polimorfismo é o princípio pelo qual duas ou mais classes derivadas de uma mesma superclasse podem invocar métodos que têm a mesma identificação, assinatura, mas comportamentos distintos, especializados para cada classe derivada, usando para tanto uma referência a um objeto do tipo da superclasse.
			
			%			\item 
			
		\end{itemize}
	}\par
	\vspace{1em}
	
\end{frame}






\begin{frame}[t]{Polimorfismo em classe}	
	
	\lstinputlisting[style=CBruno,caption=Polimorfismo de classe]{outros/codigos/python/exemplos-de-aulas/src/codigo_005_classe_produto_perecivel.py}
	
	\vspace{1em}
	
	*obs: \textbf{class ProdutoPerecivel(Produto)} é uma classe derivada de Produto: indica que \textbf{ProdutoPerecivel} é uma subclasse da classe \textbf{Produto}.
	
\end{frame}

\begin{frame}[t]{Pytest}
	
	%	\vspace{-2em}
	\lstinputlisting[style=CBruno,caption=Cobertura de testes da classe Produto Perecivel]{outros/codigos/python/exemplos-de-aulas/tests/test_codigo_005_classe_produto_perecivel.py}
	
	
\end{frame}




\begin{frame}[t]{Pytest}
	
	\centering
	\includegraphics[scale=0.5]{imagens/fig-result-test-especifico-produto-perecivel.png}
	
\end{frame}


\begin{frame}[t]{Pytest}
	
	\centering
	\includegraphics[scale=0.3]{imagens/fig-result-test-produto-perecivel.png}
	
\end{frame}






\begin{frame}[t]{Polimorfismo em classe}	
	
	\fontsize{14pt}{15}\selectfont{
		
		Sobreposição.
		
	}\par
	\vspace{1em}
	
	\fontsize{12pt}{15}\selectfont{
		\begin{itemize}%[<+->] 
			
			%			\item Sobrecarga (Overload): é o ato de criar vários métodos diferentes com o mesmo o nome, porém com assinaturas diferentes, cada um com sua própria implementação. Python não trabalha com overload por padrão. Para alcançar é possível utilizar bibliotecas com decoratores.
			
			\item Sobreposição (Override): é sobrescrever, ou seja, definir um novo comportamento para um método que já existe. Isso acontece quando a classe em questão herda (estende) outra classe e se cria um método com a mesma assinatura da classe "pai" na classe filha.
			
		\end{itemize}
	}\par
	\vspace{1em}
	
\end{frame}



\begin{frame}[t]{Encapsulamento em classe}
	
	\fontsize{12pt}{15}\selectfont{
		
		Encapsulamento é a proteção dos atributos ou métodos de uma classe, em Python existem somente o public e o private e eles são definidos no próprio nome do atributo ou método.
		
	}\par
	
	\vspace{0.5em}
	\fontsize{8pt}{10}\selectfont{
		\begin{beamercolorbox}[wd=\textwidth]{warning}
			class Veiculo:\\
			\hspace{1em}chassi = 1 \# atributo publico\\
			\hspace{1em}\_\_motor = 2 \# atributo privado a classe Veiculo. O símbolo \_\_* define como privado.\\
			\vspace{0.5em}
			class Carro(Veiculo):\\
			\hspace{1em}\_\_placa = 3 \# atributo privado a classe Carro\\
			\vspace{0.5em}
			\hspace{1em}def \_\_init\_\_(self):\\
			\hspace{2em}print(self.chassi)\\
			\hspace{2em}print(self.\_\_placa)\\
			\vspace{0.5em}
			veiculo = Veiculo()\\
			print(veiculo.chassi) \# imprime 1\\
			\vspace{0.5em}
			carro = Carro() \# Erro\\
			\# print(carro.\_\_motor) \# Erro, pois \_\_motor é privado a classe Veiculo.\\
			\# print(carro.\_\_placa) \# Erro, \_\_placa é um atributo privado, somente chamado pela classe Carro.\\
		\end{beamercolorbox}
		
	}\par
	\vspace{1em}
	
	
\end{frame}






\begin{frame}[t]{Encapsulamento em classe}
	
	\begin{block}{Exemplo anteior}		
		class Veiculo:\\
		\hspace{1em}chassi = 1 \# atributo publico\\
		\hspace{1em}\_\_motor = 2 \# atributo privado a classe Veiculo. O símbolo \_\_* define como privado.\\
		\vspace{0.5em}
		class Carro(Veiculo):\\
		\hspace{1em}\_\_placa = 3 \# atributo privado a classe Carro\\
		\vspace{0.5em}
		\hspace{1em}def \_\_init\_\_(self):\\
		\hspace{2em}print(self.chassi)\\
		\hspace{2em}print(self.\_\_placa)\\
		\vspace{0.5em}
		veiculo = Veiculo()\\
		print(veiculo.chassi) \# imprime 1\\
		\vspace{0.5em}
		carro = Carro() \# Erro\\
		\# print(carro.\_\_motor) \# Erro, pois \_\_motor é privado a classe Veiculo.\\
		\# print(carro.\_\_placa) \# Erro, \_\_placa é um atributo privado, somente chamado pela classe Carro.\\
	\end{block}
	
	
\end{frame}






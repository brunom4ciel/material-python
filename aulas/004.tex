\pgfplotstablegetelem{\thepart}{[index]\columnIndex}\of{\cronograma}
\part{\pgfplotsretval}
\label{part:\thepart}
\frame{\partpage}


\begin{frame}[t]{Padrões de desenvolvimento de software}
	\fontsize{14pt}{15.2}\selectfont{
		Para que servem e quando usar os design patterns? 
		
	}\par
	\vspace{1em}
	
	\fontsize{12pt}{15}\selectfont{
		\begin{itemize}%[<+->]  
			\item O principal objetivo dos padrões de design em desenvolvimento é deixar o código mais fácil de ser mantido e testado.
			
			
		\end{itemize}
	}\par
	\vspace{1em}
	
\end{frame}




\begin{frame}[t]{Padrões de desenvolvimento de software}
	\fontsize{14pt}{15.2}\selectfont{
		Quais as vantagens dos padrões de design em desenvolvimento?
		
	}\par
	\vspace{1em}
	
	\fontsize{12pt}{15}\selectfont{
		\begin{itemize}%[<+->]  
			\item Mesmo que os problemas de um projeto para outro não sejam iguais, há similaridades. Por focar na reutilização de soluções já testadas e aprovadas, os padrões de design oferecem maior agilidade e flexibilidade ao dia a dia dos desenvolvedores. 
			
		\end{itemize}
	}\par
	\vspace{1em}
	
\end{frame}




\begin{frame}[t]{Padrões de desenvolvimento de software}
	\fontsize{14pt}{15.2}\selectfont{
		Se os padrões de design ajudam a resolver problemas recorrentes em um software a partir de um modelo, é importante conhecer os três grupos. São eles: Creational (Criação), Structural (Estrutura) e Behavioral (Comportamental).
		
	}\par
	\vspace{1em}
	
	\fontsize{12pt}{15}\selectfont{
		\begin{itemize}%[<+->]  
			\item \textbf{Criacional}: os padrões de criação priorizam a interface e, por isso, lidam com a criação de objetos. 
			
			\item \textbf{Estrutural}: os padrões estruturais envolvem a relação entre os objetos, ou seja, a forma como eles interagem entre si
			
			\item \textbf{comportamental}: os padrões comportamentais estão diretamente ligados à comunicação entre os objetos. 
			
		\end{itemize}
	}\par
	\vspace{1em}
	
\end{frame}



\begin{frame}[t]{Exercícios}
	
	\fontsize{10pt}{15}\selectfont{
		\begin{itemize}%[<+->]  
			
			\item \glsfirst{exercicio_005}: \glsdesc{exercicio_005}
			
			\item \glsfirst{exercicio_006}: \glsdesc{exercicio_006}
			
			\item \glsfirst{exercicio_007}: \glsdesc{exercicio_007}
			
			\item \glsfirst{exercicio_008}: \glsdesc{exercicio_008}
			
		\end{itemize}
	}\par
	\vspace{1em}
	
\end{frame}


\begin{frame}[t]{Exercícios}
	
	\fontsize{10pt}{15}\selectfont{
		\begin{itemize}%[<+->]  
			
			\item \glsfirst{exercicio_009}: \glsdesc{exercicio_009}
			
			\item \glsfirst{exercicio_010}: \glsdesc{exercicio_010}
			
			\item \glsfirst{exercicio_011}: \glsdesc{exercicio_011}
			
			\item \glsfirst{exercicio_012}: \glsdesc{exercicio_012}
			
			
		\end{itemize}
	}\par
	\vspace{1em}
	
\end{frame}

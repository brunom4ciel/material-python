\pgfplotstablegetelem{\thepart}{[index]\columnIndex}\of{\cronograma}
\part{\pgfplotsretval}
\label{part:\thepart}
\frame{\partpage}




\begin{frame}[t]{Padrões de desenvolvimento de software}
	
	\fontsize{12pt}{15.2}\selectfont{
		...continuando Padrões de desenvolvimento de software.\\ 
	}\par
	\vspace{1em}
	
	
	\fontsize{12pt}{15}\selectfont{
		\begin{itemize}%[<+->]  
			
			\item {\color{red}Padrões de Criação.}
			
				\begin{itemize}%[<+->]
					\item {\color{blue}Singleton \CheckmarkBold}
					\item Abstract Factory \XSolidBrush
					\item Builder \XSolidBrush
					\item Factory Method \XSolidBrush
					\item Prototype \XSolidBrush
				\end{itemize}
				
		\end{itemize}
	}\par
	\vspace{1em}
	
	
\end{frame}




\begin{frame}[t]{Padrões de criação}

	\fontsize{12pt}{15.2}\selectfont{
		Abstract Factory \CheckmarkBold

	}\par
	\vspace{1em}


	\fontsize{12pt}{15}\selectfont{
		\begin{itemize}%[<+->]  

			\item fornece uma interface para criar famílias de objetos relacionados ou dependentes sem especificar suas classes concretas.

			\item Em outras palavras, ele permite que você crie objetos sem precisar saber exatamente qual classe concreta será instanciada.

		\end{itemize}
	}\par
	\vspace{1em}

\end{frame}



\begin{frame}[t]{Padrões de criação - Abstract Factory}
	
	\fontsize{12pt}{15.2}\selectfont{
		Principais considerações:

	}\par
	\vspace{0.5em}


	\fontsize{12pt}{14}\selectfont{
		\begin{enumerate}%[<+->]
			
			\item \textbf{AbstractFactory}: Define uma interface para criar produtos abstratos. Normalmente, essa interface declara métodos para criar cada tipo de produto que a fábrica pode produzir.
			
			\item \textbf{ConcreteFactory}: Implementa a interface da fábrica abstrata e produz instâncias concretas dos produtos.
			
			\item \textbf{AbstractProduct}: Declara uma interface para um tipo de produto. Cada produto específico deve implementar essa interface.
			
			\item \textbf{ConcreteProduct}: Implementa a interface do produto abstrato. Cada fábrica concreta criará instâncias desse tipo de produto.
			
			\item \textbf{Client}: Usa apenas as interfaces definidas pela fábrica abstrata e pelos produtos abstratos para trabalhar com os objetos. O cliente não precisa saber quais classes concretas estão sendo usadas.
			
		\end{enumerate}
	}\par
	\vspace{1em}

\end{frame}





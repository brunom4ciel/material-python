\pgfplotstablegetelem{\thepart}{[index]\columnIndex}\of{\cronograma}
\part{\pgfplotsretval}
\label{part:\thepart}
\frame{\partpage}


\begin{frame}[t]{Programação Orientada a Objetos (POO)}
	\fontsize{14pt}{15.2}\selectfont{
		Tradicionalmente a programação de sistemas considera os dados separados das funções. 

	}\par
	\vspace{1em}
	
	\fontsize{14pt}{15}\selectfont{
		\begin{itemize}%[<+->]  
			\item Os dados são	estruturados de modo a facilitar a sua manipulação pelas funções, mas as funções estão livres para usar os dados como quiserem. 
			
			\item Os aspectos segurança e integridade dos dados ficam de certa forma vulneráveis. 
			
			\item A programação tradicional de sistemas tem o seu mais forte e bem-sucedido modelo na \textbf{programação estruturada}.
			
		\end{itemize}
	}\par
	\vspace{1em}
	
\end{frame}



\begin{frame}[t]{Programação Orientada a Objetos (POO)}
	\fontsize{14pt}{15.2}\selectfont{
		Por outro lado, na programação orientada a objetos (POO), os dados específicos do objeto são estruturados junto com as funções que operam sobre esses dados. 
		
	}\par
	\vspace{1em}
	
	\fontsize{13pt}{15}\selectfont{
		\begin{itemize}%[<+->]  
			\item Linguagem como Python adota esse paradigma, onde os objetos são instâncias de classes, constituídas por variáveis (atributos) e métodos (funções) que operam sobre esses dados. 
			
			\item A POO busca robustez, adaptabilidade e reusabilidade, e seus princípios incluem modularidade, abstração e encapsulamento. 
			
			\item No desenvolvimento de software, o projeto, a implementação e os testes são etapas essenciais, e a definição clara das classes e suas responsabilidades é fundamental para o sucesso do sistema.
			
		\end{itemize}
	}\par
	\vspace{1em}
	
\end{frame}




\begin{frame}[t]{Programação Orientada a Objetos (POO)}
	
	\vspace{1em}
	\fontsize{14pt}{15}\selectfont{
		
		Na \acrfull{poo}, os dados específicos do objeto são estruturados juntamente com as funções que são permitidas sobre esses dados. Essa forma de programar é vista na linguagem Java.
		
	}\par
	\vspace{1em}
	
	
\end{frame}


\begin{frame}[t]{Programação Orientada a Objetos (POO)}
	
	\fontsize{14pt}{15}\selectfont{
		
		Os principais elementos da \acrshort{poo} são os objetos. Dizemos que um objeto é uma instância de uma classe. Uma \textbf{Classe} é constituída por variáveis (membros de dados) e métodos ou funções (membros da função).
		
	}\par
	\vspace{2em}
	
	\fontsize{14pt}{25}\selectfont{
		\begin{itemize}%[<+->]  
			\item A Classe é o modelo. Um objeto é um elemento deste modelo.
			
			\item As variáveis de uma Classe são também chamadas de \textbf{Atributos}.
			
			\item As funções de uma Classe são também chamadas de \textbf{Métodos}.
			
		\end{itemize}
	}\par
	\vspace{1em}
	
	
\end{frame}



